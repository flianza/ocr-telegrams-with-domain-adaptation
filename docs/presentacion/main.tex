\documentclass{beamer}
\usetheme{sintef}

\usefonttheme{serif}
\usepackage{amsmath,amsthm,amssymb,amsfonts,oldgerm}
\usepackage[T1]{fontenc}
\usepackage{mathpazo}

\newcommand{\testcolor}[1]{\colorbox{#1}{\textcolor{#1}{test}}~\texttt{#1}}

\titlebackground*{images/background}

\newcommand{\hrefcol}[2]{\textcolor{cyan}{\href{#1}{#2}}}

\title{Clasificación de los dígitos escritos en los telegramas de las elecciones legislativas en Santa Fe mediante técnicas de adaptación de dominio.}
\course{Maestría en Explotación de Datos y Gestión del Conocimiento}
\author{Franco \textsc{Lianza}}

\begin{document}
\maketitle

\section{Introducción}

\begin{frame}{Beamer for SINTEF slides}
      \begin{itemize}
            \item We assume you can use \LaTeX; if you cannot, \hrefcol{http://en.wikibooks.org/wiki/LaTeX/}{you can learn it here}
            \item Beamer is one of the most popular and powerful document classes for presentations in \LaTeX
            \item Beamer has also a detailed
                  \hrefcol{http://www.ctan.org/tex-archive/macros/latex/contrib/beamer/doc/beameruserguide.pdf}{user manual}
            \item Here we will present only the most basic features to get you up to speed
      \end{itemize}
\end{frame}

\section{Marco Teórico}
\section{Metodología}
\section{Análisis de Resultados}
\section{Conclusiones}

\backmatter
\end{document}
