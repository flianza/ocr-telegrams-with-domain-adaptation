\documentclass[twoside]{report}
\usepackage[utf8]{inputenc}
\usepackage{graphicx}
\usepackage{lipsum}
\graphicspath{{../images/}}

\usepackage[a4paper,width=150mm,top=25mm,bottom=25mm,bindingoffset=6mm]{geometry}
\usepackage{fancyhdr}
\pagestyle{fancy}
\fancyhf{}
\fancyhead{}
\fancyhead[RO,LE]{Thesis Title}
\fancyfoot{}
\fancyfoot[LE,RO]{\thepage}
\fancyfoot[LO,CE]{Chapter \thechapter}
\fancyfoot[CO,RE]{Author Name}
\fancypagestyle{plain}{}

\title{
	{Thesis Title}\\
	{\large Universidad Austral}\\
	{\includegraphics[width=3in,height=3in]{university.jpg}}
}
\author{Franco Lianza}
\date{Day Month Year}

\begin{document}

\maketitle

\subsection*{Tema}
{\it Título del trabajo}

Manejo del sesgo en los datos: Detecci\'on de los digitos escritos en los
telegramas de las elecciones legislativas en Santa Fe mediante t\'ecnicas de
adaptaci\'on de dominio.

\subsection*{Resumen}
{\it Resumen del área sobre la que se realizará el trabajo.}

La rama de {\it Computer Vision} se ha extendido exponencialmente a lo largo
del tiempo. Se ha llegado a un punto en el cual se pueden detectar todo tipo de
objetos con una precisi\'on m\'as que \'optima.

Las t\'ecnicas desarrolladas en el \'area precisan de un gran volumen de datos
para su entrenamiento. Esto implica que es de suma importancia de tener
disponibles las etiquetas (o {\it lables}) de los datos que se van a utilizar
para entrenar los modelos. El etiquetado de los datos es una tarea costosa,
ineficiente y hasta a veces resulta inviable de realizar.

A\'un teniendo el etiquetado de los datos, puede ocurrir que el conjunto de
datos ({\it dataset} en ingl\'es) donde se va a utilizar el modelo resulte
diferente al que se utilizó para entrenarlo. Por mencionar, un modelo de
detecci\'on de rostros entrenado en una etnia demogr\'afica particular
funcionar\'a de manera err\'onea si se lo aplica a otra. A esto se lo conoce
como {\it sesgo en los datos} (tambien {\it dataset bias} o {\it dataset shift}
en ingl\'es).

El {\it sesgo en los datos} es un problema que ocurre en la mayor\'ia de los
conjuntos de datos con los que se entrenan los modelos. De hecho, algunos
autores afirman que el sesgo es un problema que no se puede evitar (ref? ) al
crear un dataset.

\subsection*{Director o Tutor}
{\it El nombre del director o tutor de la tesis o trabajo, si éste ya
	hubiese aceptado la tarea.}

TODO:

\subsection*{Motivación e importancia del campo}
{\it Explicar la o las motivaciones que llevan a realizar el trabajo planteado y su importancia.}

\lipsum[1]

\subsection*{Problemas no resueltos}
{\it Problemas no resueltos detectados en el área y que el trabajo a realizar pretende resolver.}

\lipsum[1]

\subsection*{Objetivo del trabajo}
{\it Explicar claramente el objetivo del trabajo, especificando su alcance y limitaciones.}

\lipsum[1]

\subsection*{Requerimientos y desafíos}
{\it Requerimientos y desafíos que plantea el trabajo a realizar.}

\lipsum[1]

\subsection*{Metodología}
{\it Metodología a emplear para el desarrollo del trabajo.}

\lipsum[1]

\subsection*{Plan de Trabajo}
{\it Especificar las distintas tareas a realizar con los tiempos que se
	estime que deberían insumir.
	Indicar las fechas estimadas de inicio y finalización del trabajo.
}

\lipsum[1]

\subsection*{Bibliografía}
{\it Listar las referencias bibliográficas del material utilizado para la definición del
	trabajo a realizar, así como también las que se pueden llegar emplear en la
	concreción del mismo}

\lipsum[1]

\end{document}