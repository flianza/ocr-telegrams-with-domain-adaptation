\documentclass[a4paper, twoside]{report}
\usepackage[utf8]{inputenc}
\usepackage{graphicx}
\usepackage[colorlinks=true, allcolors=blue]{hyperref}
\graphicspath{{images/}}

\usepackage[a4paper,width=150mm,top=25mm,bottom=25mm,bindingoffset=6mm]{geometry}
\usepackage{fancyhdr}
\pagestyle{fancy}
\fancyhf{}
\fancyhead{}
\fancyhead[RO,LE]{Proyecto de Tesis}
\fancyfoot{}
\fancyfoot[LE,RO]{\thepage}
\fancypagestyle{plain}{}

\usepackage{amsmath}
\usepackage[colorinlistoftodos]{todonotes}
\usepackage{pgfgantt}
\renewcommand{\figurename}{Figura}

\title{Proyecto de Tesis\\ \large Clasificaci\'on de los digitos escritos en los telegramas de las elecciones legislativas en Santa Fe mediante
t\'ecnicas de adaptaci\'on de dominio.}
\author{Franco Lianza}
\begin{document}

\begin{titlepage}

    \newcommand{\HRule}{\rule{\linewidth}{0.5mm}}

    \center
    \includegraphics[width=0.4\textwidth]{university.jpg}\\[1cm]

    \textsc{\Large Maestr\'ia en Explotaci\'on de Datos y Gesti\'on del Conocimiento}\\[1.5cm]

    \makeatletter
    \HRule \\[0.6cm]
    {\huge \bfseries \@title}\\[0.4cm]
    \HRule \\[2cm]

    {\Large Autor: \@author}\\[1cm]
    {\Large Tutor: Dr. Leandro Bugnon}\\[3cm]

    {\large Julio 2022}\\[2cm]

    \vfill
\end{titlepage}
\newpage

\section*{Resumen}

En Argentina se celebran elecciones cada 2 a\~{n}os, lo cual implica que se debe incurrir en altos costos para poder
llevarlas a cabo. Al finalizar el dia del sufragio, cada una de las mesas de voto arman una planilla con los votos
obtenidos por cada partido pol\'itico. La misma es enviada al correo en forma de telegrama para luego ser escaneado y
enviado a un centro de c\'omputo. El proceso por el cual se digitalizan a mano cada uno de los telegramas los
telegramas en un sistema central es costoso, ineficiente y suceptible a errores humanos. Contar con una herramienta que
permita digitalizar autom\'aticamente cada uno de los telegramas al sistema central de c\'omputo, agilizar\'a y
aportar\'a transparencia al proceso reduciendo los costos y errores. Por tal motivo, se centrar\'a en desarrollar un
proceso de extracci\'on de d\'igitos de los telegramas para luego entrenar distintos modelos que sean capaces de
clasificarlos. De esta manera, se podr\'an definir m\'etricas que permitan seleccionar el mejor de ellos y
disponibilizarlo en las pr\'oximas elecciones.

\section*{Motivación e importancia del campo}

Existen, principalmente, tres tipos de elecciones:

\begin{itemize}
	\item Elecciones nacionales, para elegir a las autoridades federales del país: el Poder Ejecutivo, constituido por el
	      Presidente y el vicepresidente y el Congreso Nacional, formado por Senadores y Diputados.
	\item Elecciones provinciales y de la Ciudad de Buenos Aires o locales, para elegir a las autoridades de cada provincia: los
	      poderes ejecutivos de las provincias y sus legislaturas.
	\item Elecciones municipales, regidas por las leyes y procedimientos de cada provincia.
\end{itemize}

Si bien emitir el sufragio es diferente en cada una de ellas, generalmente consta de ingresar a un cuarto oscuro,
elegir el candidato que se desea y depositar el voto en una urna. Al finalizar la jornada, las autoridades de mesas
recuentan los votos y llenan una planilla a mano alzada donde se resume la cantidad de votos obtenidos por cada
candidato o partido pol\'itico. Dicha planilla es escaneada y enviada a traves de un telegrama del correo argentino al
centro de c\'omputo para su procesamiento. Una vez all\'i, se contabilizan en un sistema inform\'atico a partir de un
grupo de empleados. Finalmente, se procede a una etapa de validaci\'on de los datos para constatar que lo computado
coincide con lo escrito.

Para que el proceso sea lo mas r\'apido y eficaz posible, se requiere a una gran cantidad personas destinadas al centro
de c\'omputo. Tal soluci\'on hace que el proceso sea altamente ineficiente en cuanto a tiempos y costos se refiere. En
las elecciones legislativas del 2021 se gastaron unos \$17.000 millones de pesos de los cuales \$4.000 millones de
pesos fueron destinados a sueldos para el personal\footnote{Fuente:
	\href{https://www.cronista.com/economia-politica/Elecciones-legislativas-2021-cuanto-mas-se-gastara-por-el-coronavirus-segun-el-Presupuesto-20201004-0006.html}{El
		cronista}}.

\section*{Requerimientos y desafíos}

La clasificaci\'on de n\'umeros es un problema que, si bien parece resuelto con \cite{lecun1998gradient} y la
creaci\'on del dataset MNIST, no debe ser tomada a la ligera. Si bien la rama de {\it Computer Vision}, que se encarga
de desarrollar herramientas capaces de reconocer patrones complejos en im\'agenes, se encuentra en un punto en el cual
presenta resultados \'optimos en la mayor\'ia de los casos \cite{szeliski2010computer, redmon2016yolo}; no se encuentra
exenta al problema del sesgo en los datos.

No existe una \'unica forma de escribir y a\~{n}o a a\~{n}o cambian las personas que son los jefes de mesa encargados
de completar los telegramas. Las caracter\'isticas de los n\'umeros escritos difiere entre cada elecci\'on. Cuando la
distribuci\'on de los datos de entrenamiento difiere a la de los datos de aplicaci\'on, se est\'a ante un {\it
		corrimiento de dominio} ({\it domain shift} o {\it data drift} en ingl\'es). Esto provoca los siguientes problemas:

\begin{itemize}
	\item No se puede utilizar un modelo entrenado en telegramas de elecciones pasadas.
	\item Para entrenar alg\'un modelo de clasificaci\'on de im\'agenes se requiere de una gran cantidad de datos y sus
	      respectivas etiquetas.
\end{itemize}

Es por lo expuesto que no puede utilizarse un esquema de entrenamiento convencional. Una alternativa es la realizada en
trabajos anteriores, donde se aplicaron distorsiones al conjunto de entrenamiento para aumentar la cantidad de datos de
entrenamiento y que de esta forma el modelo pueda generalizar y aplicarse a los telegramas de elecciones de la Ciudad
de Buenos Aires \cite{lamagna2016lectura}. No obstante, no se muestran m\'etricas concretas respecto de los resultados
obtenidos en los telegramas reales.

\section*{Problemas no resueltos}

El problema que quiere resolver es la digitalizaci\'on autom\'atica de los telegramas mediante machine learning de
forma robusta.

La digitalizaci\'on de los telegramas de las elecciones sigue siendo de forma manual, lo que genera lentitud y
desconfianza en el proceso. Detectar los votos de manera autom\'atica y reducir la intervenci\'on humana, aumentar\'a
la eficiencia de las elecciones y la confianza en ellas.

\section*{Soluci\'on propuesta}

Durante el proceso de digitalizaci\'on de los telegramas en una elecci\'on, se ejecutar\'a el proceso que se encarga de
extraer los d\'igitos de los mismos y entrenar\'a un modelo de clasificaci\'on espec\'ifico a partir de {\it transfer
		learning}. Luego, se aplicar\'a el modelo y su salida ser\'a el voto digitalizado junto a su probabilidad. De esta
menera, la digitalizaci\'on estar\'a realizada y se podra\'a proceder al proceso de validaci\'on humana de las
predicciones.

El {\it transfer learning} (transferencia de aprendizaje) es un \'area del {\it machine learning} que se encarga de
almacenar el conocimiento ganado en un problema y aplicarlo a otro \cite{thrun1998learning}. Sin embargo, como no se
dispondr\'an de las etiquetas al momento de la elecci\'on, se utilizar\'an t\'ecnicas de {\it adaptaci\'on de dominio}
(AD). La misma consiste en la habilidad de aplicar un modelo entrenado en uno o mas dominios de origen ({\it source
		domain}) en un uno distinto pero relacionado ({\it target domain}) \cite{ben2006analysis}. A modo de ejemplo, la figura
\ref{fig:mnist-to-svhn} muestra dos conjuntos de datos de d\'igitos pero con dominios diferentes.

\begin{figure}[ht]
	\centering
	\includegraphics[width=0.6\textwidth]{mnist-to-svhn.png}
	\caption{Ejemplo dominios diferentes: MNIST y SVHN}
	\label{fig:mnist-to-svhn}
\end{figure}

\section*{Objetivo del trabajo}

Como objetivo general del trabajo se pretende desarrollar un proceso que permita transformar los telegramas de
elecciones legislativas de Santa Fe, entrenar un modelo de clasificaci\'on mediante una t\'ecnica de adaptaci\'on de
dominio y poder digitalizar los telegramas.

Esto desprende los siguientes objetivos espec\'ificos:
\begin{itemize}
	\item Armar el proceso de ETL de los telegramas que permita limpiar y extraer los d\'igitos.
	\item Determinar un conjunto de datos etiquetado a ser utilizado como dominio de origen.
	\item Entrenar distintos arquitecturas de redes convolucionales mediante t\'ecnicas de adaptaci\'on de dominio.
	\item Analizar y evaluar las m\'etricas que permitan seleccionar el mejor modelo.
	\item Seleccionar el mejor par modelo - t\'ecnica AD.
\end{itemize}

\section*{Metodología}

Se har\'a revisi\'on del estado del arte para despu\'es continuar con el preprocesamiento de los datos. Los telegramas
ser\'an descargados desde la \href{https://op.elecciones.gob.ar/telegramas/generales2021/}{p\'agina oficial del estado
	argentino}. Luego se extraer\'an los d\'igitos de los votos de cada telegrama utilizando la librer\'ia {\it OpenCV}
\cite{opencv_library} y Python. Se realizar\'an ciclos de entrenamiento de distintas redes convolucionales (LeNet
\cite{lecun1998gradient}, ResNet \cite{he2016deep}, etc) en el dominio de origen para luego aplicar t\'ecnicas de AD
que permita transferir el conocimiento al dominio de destino (telegramas legislativos de Santa Fe). Se evaluar\'an las
m\'etricas de cada uno de ellos y se proceder\'a a elegir el mejor.

\section*{Plan de Trabajo}

El trabajo se realizar\'a a partir de las siguientes etapas:
\begin{itemize}
	\item \underline{Estudio del estado del arte}: estudiar las diferentes t\'ecnicas
	      de {\it domain adaptation}.
	\item \underline{Extracci\'on datos}: recolectar los telegramas
	      y extraer los d\'igitos de los votos.
	\item \underline{Limpieza de datos}: detectar votos incorrectos (por ejemplo
	      letras) y tratarlos para reducir lo m\'aximo posible el ruido
	      en los datos.
	\item \underline{Experimentaci\'on}: realizar ciclos de entrenamiento de modelos
	      y adaptarlos mediante alguna de las t\'ecnicas disponibles.
	\item \underline{Elaboraci\'on de reportes}: generar reportes que sinteticen los
	      experimentos realizados para compararlos.
	\item \underline{Redacci\'on de tesis}.
\end{itemize}

Se estima que el proyecto comienza en Agosto del 2022 y finaliza en Diciembre del 2022. La figura \ref{fig:gantt}
presenta el cronograma propuesto.

\begin{figure}[ht]
	\begin{center}

		\begin{ganttchart}[y unit title=0.4cm,
				y unit chart=0.5cm,
				vgrid,hgrid,
				title label anchor/.style={below=-1.6ex},
				title left shift=.05,
				title right shift=-.05,
				title height=1,
				progress label text={},
				bar height=0.7,
				group right shift=0,
				group top shift=.6,
				group height=.3]{1}{20}
			%labels
			\gantttitle{Agosto}{4}
			\gantttitle{Septiembre}{4}
			\gantttitle{Octubre}{4}
			\gantttitle{Noviembre}{4}
			\gantttitle{Diciembre}{4}\\
			\gantttitle{1}{1}
			\gantttitle{2}{1}
			\gantttitle{3}{1}
			\gantttitle{4}{1}
			\gantttitle{1}{1}
			\gantttitle{2}{1}
			\gantttitle{3}{1}
			\gantttitle{4}{1}
			\gantttitle{1}{1}
			\gantttitle{2}{1}
			\gantttitle{3}{1}
			\gantttitle{4}{1}
			\gantttitle{1}{1}
			\gantttitle{2}{1}
			\gantttitle{3}{1}
			\gantttitle{4}{1}
			\gantttitle{1}{1}
			\gantttitle{2}{1}
			\gantttitle{3}{1}
			\gantttitle{4}{1}\\

			%tasks
			\ganttbar{Estudio del estado del arte}{1}{6} \\
			\ganttbar{Extracci\'on de datos}{3}{4} \\
			\ganttbar{Limpieza de datos}{5}{6} \\
			\ganttbar{Esperimentaci\'on}{7}{12} \\
			\ganttbar{Elaboraci\'on de reportes}{12}{13} \\
			\ganttbar{Redacci\'on de tesis}{14}{20}

		\end{ganttchart}
	\end{center}
	\caption{Distribuci\'on de las tareas del proyecto.}
	\label{fig:gantt}
\end{figure}

\section*{Bibliograf\'ia}
 {
  %Disable section* command
  \renewcommand{\section*}[2]{}
  \bibliographystyle{unsrt}
  \bibliography{refs}
 }

\end{document}