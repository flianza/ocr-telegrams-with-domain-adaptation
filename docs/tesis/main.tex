%----------------------------------------------------------------------------------------
%	PACKAGES AND OTHER DOCUMENT CONFIGURATIONS
%----------------------------------------------------------------------------------------

\documentclass[
12pt, % The default document font size, options: 10pt, 11pt, 12pt
%oneside, % Two side (alternating margins) for binding by default, uncomment to switch to one side
spanish, % ngerman for German
singlespacing, % Single line spacing, alternatives: onehalfspacing or doublespacing
%draft, % Uncomment to enable draft mode (no pictures, no links, overfull hboxes indicated)
%nolistspacing, % If the document is onehalfspacing or doublespacing, uncomment this to set spacing in lists to single
%liststotoc, % Uncomment to add the list of figures/tables/etc to the table of contents
%toctotoc, % Uncomment to add the main table of contents to the table of contents
%parskip, % Uncomment to add space between paragraphs
%nohyperref, % Uncomment to not load the hyperref package
headsepline, % Uncomment to get a line under the header
%chapterinoneline, % Uncomment to place the chapter title next to the number on one line
%consistentlayout, % Uncomment to change the layout of the declaration, abstract and acknowledgements pages to match the default layout
]{MastersDoctoralThesis} % The class file specifying the document structure

\usepackage[utf8]{inputenc} % Required for inputting international characters
\usepackage[T1]{fontenc} % Output font encoding for international characters
\usepackage{mathpazo}
\usepackage[backend=bibtex,style=authoryear,natbib=true]{biblatex} % Use the bibtex backend with the authoryear citation style (which resembles APA)
\usepackage{lipsum}
\usepackage{amsmath}
\usepackage{caption}
\usepackage{subcaption}
\usepackage{algorithm}
\usepackage{algorithmic}
\usepackage{amssymb}

\addbibresource{bibliography.bib}

\usepackage[autostyle=true]{csquotes} % Required to generate language-dependent quotes in the bibliography
\usepackage{tikz}
\graphicspath{{images/}}
\usepackage{booktabs}
\usepackage{multirow}
\usepackage{float}
\usepackage{mathtools}
\usepackage{diagbox}

\floatstyle{plaintop}
\restylefloat{table}

\usetikzlibrary{matrix}
\usetikzlibrary{positioning}
\usetikzlibrary{backgrounds}

\newcommand\numRowsK{3}
\newcommand\numColsK{3}
\newcommand{\K}[2]{% #1: row, #2: col
    \edef\Kcol##1##2##3{###2}%
    \edef\Krow##1##2##3{\noexpand\Kcol###1}%
    \Krow
        {1 0 1}
        {0 1 0}
        {1 0 1}%
}

\newcommand{\convoutionpicture}[2]{% #1: row to be highlighted, #2: colum to be highlighted
\begin{tikzpicture}
    % ------- style -------
    \tikzset{%
        parenthesized/.style={%
            left delimiter  = (,
            right delimiter = ),
        },
        node distance = 10mu,
    }

    % ------- equation -------
    \matrix[matrix of math nodes, parenthesized, ampersand replacement=\&] (I) {
        0 \& 1 \& 1 \& 1 \& 0 \& 0 \\
        0 \& 0 \& 1 \& 1 \& 1 \& 0 \\
        0 \& 1 \& 0 \& 1 \& 0 \& 0 \\
        0 \& 0 \& 1 \& 1 \& 0 \& 0 \\
        0 \& 1 \& 1 \& 0 \& 0 \& 0 \\
        1 \& 0 \& 0 \& 0 \& 0 \& 0 \\
    };

    \node (*) [right = of I] {${}*{}$};

    \def\Kmatrix{}
    \foreach \row in {1, ..., 3} {
        \gdef \sep {}
        \foreach \col in {1, ..., 3} {%
            \xdef \Kmatrix {\unexpanded\expandafter{\Kmatrix}\unexpanded\expandafter{\sep}\noexpand \K{\row}{\col}}
            \gdef \sep { \& }
        }
        \xdef \Kmatrix {\unexpanded\expandafter{\Kmatrix}\noexpand\\}
    }
    \matrix[matrix of math nodes, parenthesized, ampersand replacement=\&] (K) [right = of *] {
        \Kmatrix
    };

    \node (=) [right = of K] {${}={}$};

    \matrix[matrix of math nodes, parenthesized, ampersand replacement=\&] (I*K) [right = of {=}] {
        1 \& 5 \& 2 \& 3 \\
        3 \& 2 \& 4 \& 2 \\
        1 \& 4 \& 2 \& 1 \\
        3 \& 2 \& 1 \& 1 \\
    };

    % ------- highlighting -------
    \def\rowResult{#1}
    \def\colResult{#2}

    \begin{scope}[on background layer]
        \newcommand{\padding}{2pt}
        \coordinate (Is-nw) at ([xshift=-\padding, yshift=+\padding] I-\rowResult-\colResult.north west);
        \coordinate (Is-se) at ([xshift=+\padding, yshift=-\padding] I-\the\numexpr\rowResult+\numRowsK-1\relax-\the\numexpr\colResult+\numColsK-1\relax.south east);
        \coordinate (Is-sw) at (Is-nw |- Is-se);
        \coordinate (Is-ne) at (Is-se |- Is-nw);

        \filldraw[red,   fill opacity=.1] (Is-nw) rectangle (Is-se);
        \filldraw[green, fill opacity=.1] (I*K-\rowResult-\colResult.north west) rectangle (I*K-\rowResult-\colResult.south east);

        \draw[blue, dotted] 
            (Is-nw) -- (K.north west)
            (Is-se) -- (K.south east)
            (Is-sw) -- (K.south west)
            (Is-ne) -- (K.north east)
        ;
        \draw[green, dotted] 
            (I*K-\rowResult-\colResult.north west) -- (K.north west)
            (I*K-\rowResult-\colResult.south east) -- (K.south east)
            (I*K-\rowResult-\colResult.south west) -- (K.south west)
            (I*K-\rowResult-\colResult.north east) -- (K.north east)
        ;

        \draw[blue,  fill=blue!10!white] (K.north west) rectangle (K.south east);

        \foreach \row [evaluate=\row as \rowI using int(\row+\rowResult-1)] in {1, ..., \numRowsK} {%
            \foreach \col [evaluate=\col as \colI using int(\col+\colResult-1)] in {1, ..., \numColsK} {%
                    \node[text=blue] at (I-\rowI-\colI.south east) [xshift=-.3em] {\tiny$\times \K{\row}{\col}$};
                }
        }
    \end{scope}

    % ------- labels -------
    \tikzset{node distance=0em}
    \node[below=of I] (I-label) {$I$};
    \node at (K |- I-label)     {$K$};
    \node at (I*K |- I-label)   {$I*K$};
\end{tikzpicture}%
}

%----------------------------------------------------------------------------------------
%	MARGIN SETTINGS
%----------------------------------------------------------------------------------------

\geometry{
	paper=a4paper, % Change to letterpaper for US letter
	inner=2.5cm, % Inner margin
	outer=3.8cm, % Outer margin
	bindingoffset=.5cm, % Binding offset
	top=1.5cm, % Top margin
	bottom=1.5cm, % Bottom margin
	%showframe, % Uncomment to show how the type block is set on the page
}

%----------------------------------------------------------------------------------------
%	THESIS INFORMATION
%----------------------------------------------------------------------------------------

\thesistitle{Clasificación de los dígitos escritos en los telegramas de las elecciones legislativas en Santa Fe mediante técnicas de adaptación de dominio.} 
\supervisor{Dr. Leandro \textsc{Bugnon}}
\examiner{}
\degree{Magíster en Explotación de Datos y Gestión del Conocimiento}
\author{Ing. Franco \textsc{Lianza}}

\keywords{} 
\university{\href{https://www.austral.edu.ar/}{Universidad Austral}}
\department{}
\group{}
\faculty{\href{https://www.austral.edu.ar/ingenieria/}{Facultad de Ingeniería}} 

\AtBeginDocument{
\hypersetup{pdftitle=\ttitle}
\hypersetup{pdfauthor=\authorname}
\hypersetup{pdfkeywords=\keywordnames}
}

\begin{document}

\frontmatter % Use roman page numbering style (i, ii, iii, iv...) for the pre-content pages

\pagestyle{plain} % Default to the plain heading style until the thesis style is called for the body content

%----------------------------------------------------------------------------------------
%	TITLE PAGE
%----------------------------------------------------------------------------------------

\begin{titlepage}
    \begin{center}

        \includegraphics[width=0.4\textwidth]{university}

        \vspace*{.06\textheight}
        \textsc{\Large Tesis de Maestría}\\[0.5cm] % Thesis type

        \HRule \\[0.4cm] % Horizontal line
        {\Large \bfseries \ttitle\par}\vspace{0.4cm} % Thesis title
        \HRule \\[1.5cm] % Horizontal line

        \begin{minipage}[t]{0.4\textwidth}
            \begin{flushleft} \large
                \emph{Autor:}\\
                \authorname
            \end{flushleft}
        \end{minipage}
        \begin{minipage}[t]{0.4\textwidth}
            \begin{flushright} \large
                \emph{Supervisor:} \\
                \supname
            \end{flushright}
        \end{minipage}\\[3cm]

        \vfill

        {\large \today}\\[4cm]

        \vfill
    \end{center}
\end{titlepage}

%----------------------------------------------------------------------------------------
%	ABSTRACT PAGE
%----------------------------------------------------------------------------------------

\begin{abstract}
    \addchaptertocentry{\abstractname} % Add the abstract to the table of contents
    TODO: Abstract?
\end{abstract}

%----------------------------------------------------------------------------------------
%	ACKNOWLEDGEMENTS
%----------------------------------------------------------------------------------------

\begin{acknowledgements}
    \addchaptertocentry{\acknowledgementname}
    TODO: reconocimientos...
\end{acknowledgements}

%----------------------------------------------------------------------------------------
%	LIST OF CONTENTS/FIGURES/TABLES PAGES
%----------------------------------------------------------------------------------------

\tableofcontents % Prints the main table of contents

\listoffigures % Prints the list of figures

\listoftables % Prints the list of tables

%----------------------------------------------------------------------------------------
%	THESIS CONTENT - CHAPTERS
%----------------------------------------------------------------------------------------

\mainmatter % Begin numeric (1,2,3...) page numbering

\pagestyle{thesis} % Return the page headers back to the "thesis" style

\chapter{Introducci\'on}

\label{Chapter1}

\section{Elecciones en Argentina}

En Argentina se celebran elecciones cada 2 a\~{n}os a excepci\'on de las presidenciales que se realizan cada 4
a\~{n}os. En Argentina se realizan tres tipos de elecciones:

\begin{itemize}
    \item Elecciones nacionales, para elegir a las autoridades federales del país: el Poder Ejecutivo, constituido por el
          Presidente y el vicepresidente y el Congreso Nacional, formado por Senadores y Diputados.
    \item Elecciones provinciales y de la Ciudad de Buenos Aires o locales, para elegir a las autoridades de cada provincia: los
          poderes ejecutivos de las provincias y sus legislaturas.
    \item Elecciones municipales, regidas por las leyes y procedimientos de cada provincia.
\end{itemize}

Si bien emitir el sufragio es diferente en cada una de ellas, generalmente consta de ingresar a un cuarto oscuro,
elegir el candidato que se desea y depositar el voto en una urna. Al finalizar la jornada, las autoridades de mesas
recuentan los votos y llenan una planilla a mano alzada donde se resume la cantidad de votos obtenidos por cada
candidato o partido pol\'itico. Dicha planilla es escaneada y enviada a traves de un telegrama correo argentino al
centro de c\'omputo para su procesamiento. Una vez all\'i, se contabilizan en un sistema inform\'atico una por una. Por
la metodolog\'ia de contabilizac\'ion, idealmente lo escrito a mano en el telegrama y lo computado en el sistema es lo
mismo. Sin embargo, como esta tarea es realizada por personas, es plausible pensar que pueden haber errores en dicho
proceso.

(TODO: ver de agregar un diagrama aca del proceso)
(TODO: ver de agregar una imagen de un telegrama en el apendice o anexo)

A su vez, durante la jornada electoral existe una ansiedad generalizada para ir sabiendo los resultados parciales y
finales de la misma, por lo que se debe contratar a una gran cantidad personas destinadas al centro de c\'omputo. En
las elecciones legislativas del 2021 se gastaron unos \$17.000 millones de pesos de los cuales \$4.000 millones de
pesos fueron destinados a sueldos para el personal\footnote{Fuente:
    \href{https://www.cronista.com/economia-politica/Elecciones-legislativas-2021-cuanto-mas-se-gastara-por-el-coronavirus-segun-el-Presupuesto-20201004-0006.html}{El
        cronista}}. Mejorar el proceso manual de contabilizaci\'on de los telegramas supondr\'a un ahorro considerable en el
presupuesto de las elecciones, agilizar\'a la obtenci\'on de los resultados y aportar\'a transparencia al proceso en
general.

(TODO: buscar alguna noticia de ver si algun pais ya digitaliza automaticamente o khe)

\section{Clasificaci\'on de d\'igitos (TODO: ???)}

La clasificaci\'on de d\'igitos escritos a mano lleva resuelto hace un tiempo con una performance \'optima.
\cite{lecun1998gradient} propone una arquitectura de red neuronal con m\'ultiples capas y clasifica correctamente el
dataset {\it MNIST} con ella. Se podr\'ia proponer un modelo similar para esto.

La detecci\'on de d\'igitos en los telegramas de elecciones en Argentina podr\'ia llevarse a cabo mediante un modelo
entrenado en {\it datasets} de d\'igitos p\'ublicos como el {\it MNIST} \cite{lecun1998gradient}. Como no existe una
\'unica forma de escribir, el modelo estar\'a sesgado a reconocer d\'igitos escritos de forma similar a los que se
encontraban en el {\it dataset} de entrenamiento. No ser\'a capaz de generalizar lo aprendido en un dominio distinto.

En trabajos anteriores, se aplican distorsiones al conjunto de entrenamiento para aumentar la cantidad de datos de
entrenamiento y de esta forma el modelo pueda generalizar y aplicarse a los telegramas de elecciones de la Ciudad de
Buenos Aires \cite{lamagna2016lectura}. En el presente trabajo se utilizar\'an t\'ecnicas referidas al {\it transfer
        learning}, espec\'ificamente de {\it domain adaptation} para resolver el problema.

\section{Deep Learning}

Cuando se habla de {\it Deep learning}, se hace referencia a una serie de algoritmos de {\it machine learning} que son
capaces de utilizar m\'ultiples capas de procesamiento de forma que puedan aprender representaciones de los datos con
diferentes niveles de abstracci\'on \parencite{lecun2015deep}. Estos algoritmos, denominados redes neuronales profundas (o DNNs por sus siglas en ingl\'es),
poseen la capacidad de encontrar variables que expliquen la naturaleza del comportamiento de los datos.

Los modelos obtenidos a partir del {\it deep learning} han demostrado tener gran capacidad de aprendizaje para todo
tipo de problemas, como ser {\it computer vision} \parencite{szeliski2010computer, redmon2016yolo}, procesamiento del lenguaje natural \parencite{devlin2018bert}, reconocimiento del habla \parencite{hannun2014deep}, juegos \parencite{silver2016mastering}, generaci\'on de im\'agenes a partir de descripciones \parencite{ramesh2022dalle2}, entre otros.

Aunque la utilidad de estos modelos se encuentra demostrada y d\'ia a d\'ia so utilizados en diferentes \'ambitos de la
vida, presentan un gran problema: la enorme cantidad de datos que requieren para su entrenamiento. La mayor\'ia de los
modelos que mejores m\'etricas de performance presentan, necesitan millones de datos en sus datasets de entrenamiento.
Esto implica que, para que los mismos sean de utilidad, es de suma importancia los procesos de recolecci\'on y
etiquetado de los datos. La eficacia de los modelos queda altamente relacionada con la calidad de los datos que se
posean o se logren conseguir. Particularme, el etiquetado de los datos es una tarea costosa, ineficiente y hasta a
veces resulta inviable de realizar \parencite{reis2022data}.

Una posible soluci\'on a este problema consiste en emular la capacidad que tienen los humanos de adquirir conocimiento
relevante en un \'area y aplicarlo en otra similar \parencite{thrun1998learning}. Es decir, poder {\it transferir} lo aprendido. En el caso del {\it deep learning}, lo que se
busca es que la red aprenda representaciones lo suficientemente generales para que despues sean utilizados en el
entrenamiento de una tarea similar. Esto busca acortar los tiempos de entrenamiento, mejorar las predicciones y hacer
los modelos m\'as robustos.

\chapter{Estado del Arte / Marco teorico?}

\label{Chapter2}

\section{Reconocimiento de d\'igitos}

\lipsum[1]

\section{Redes Neuronales}

\lipsum[1]

\section{Domain Adaptation}

\lipsum[1]
\chapter{Metodolog\'ia}

\label{Chapter3}

\section{Extracci\'ion de d\'igitos de los telegramas}

\lipsum[1]

\section{Modelo baseline}

\lipsum[1]

\section{Adaptaci\'on de dominio}

\lipsum[1]

\subsection{DANN}

\lipsum[1]

\subsection{ADDA}

\lipsum[1]

\subsection{AFN}

\lipsum[1]

\subsection{MDD}

\lipsum[1]

\chapter{An\'alisis de resultados}

\label{Chapter4}

A lo largo del presente cap\'itulo, se analizar\'an los resultados obtenidos a partir de los experimentos planteados.
Se detallar\'an las m\'etricas obtenidas y los resultados de aplicar los modelos a los telegramas.

\section{An\'alisis de m\'etricas}

Los experimentos realizados arrojaron los resultados mostrados en el cuadro \ref{tab:metricas-experimentos}. Las
m\'etricas referidas a la capacidad de clasificaci\'on (Accuracy, $F_1$) son evaluadas sobre la partici\'on de test del
dataset de origen donde se conocen las etiquetas a ciencia cierta ($MNIST$). Por otro lado, las m\'etricas de
adaptaci\'on ($MMD$, Dist. $\mathcal{A}$) son evaluadas sobre los espacios latentes que los modelos generaron para las
particiones de test de ambos datasets. El mejor modelo es el que posea mejor capacidad de clasificaci\'on y de
adaptaci\'on.

% TODO: cambiar f1 por macro para que sea distinto a acc. que onda iou?
\begin{table}[H]
    \centering
    \begin{tabular}{cc|rrrr}
        \toprule
                                 &        & Acc.                               & $F_1$                               & $MMD$                               & Dist. $\mathcal{A}$                 \\
        AD                       & Modelo &                                    &                                     &                                     &                                     \\
        \midrule
        \multirow[c]{2}{*}{-}    & ResNet & \textbf{{\footnotesize (1)} 98.73} & \textbf{{\footnotesize (1)} 0.9873} & 0.0595                              & 1.9443                              \\
                                 & LeNet  & 98.14                              & 0.9814                              & 0.0498                              & 1.9313                              \\\hline
        \multirow[c]{2}{*}{MDD}  & ResNet & 98.58                              & 0.9858                              & 0.0569                              & 1.9031                              \\
                                 & LeNet  & {\footnotesize (3)} 98.62          & {\footnotesize (3)} 0.9862          & 0.0363                              & 1.7259                              \\\hline
        \multirow[c]{2}{*}{DANN} & ResNet & 97.52                              & 0.9752                              & 0.0153                              & 1.6383                              \\
                                 & LeNet  & 98.14                              & 0.9814                              & {\footnotesize (2)} 0.0119          & {\footnotesize (3)} 1.6231          \\\hline
        \multirow[c]{2}{*}{AFN}  & ResNet & {\footnotesize (2)} 98.68          & {\footnotesize (2)} 0.9868          & \textbf{{\footnotesize (1)} 0.0037} & \textbf{{\footnotesize (1)} 1.0047} \\
                                 & LeNet  & 98.58                              & 0.9857                              & {\footnotesize (3)} 0.0138          & {\footnotesize (2)} 1.5920          \\\hline
        \multirow[c]{2}{*}{ADDA} & ResNet & 89.18                              & 0.8918                              & 0.0225                              & 1.8691                              \\
                                 & LeNet  & 97.49                              & 0.9749                              & 0.0288                              & 1.8112                              \\
        \bottomrule
    \end{tabular}
    \caption{M\'etricas de los experimentos realizados. Entre par\'entesis se encuentra la posici\'on que ocupa dentro del top 3 de la columna.}
    \label{tab:metricas-experimentos}
\end{table}

Como era de esperarse, los modelos que fueron entrenados sin adaptaci\'on de dominio son los que mejores m\'etricas de
clasificaci\'on poseen. No obstante, sus valores de adaptaci\'on son los peores provocando que no puedan ser aplicados
en $TDS$.

El modelo que mejor combinaci\'on de clasificaci\'on y adaptaci\'on es la ResNet utilizando AFN. Cabe destacar que los
modelos LeNet entrenados con DANN y AFN obtuvieron buenas m\'etricas en general, lo que deja en evidencia que {\it no
        siempre un modelo m\'as complejo es mejor}.

Es posible aplicar los modelos sobre todos los telegramas y calcular el $IoU$ promedio por telegrama con el c\'alculo
descripto en el cap\'itulo \ref{Chapter3} y la cantidad de aciertos promedio por telegrama utilizando como etiquetas lo
transcripto en el centro de c\'omputo suponiendo que existen pocos errores en ellos.

\begin{table}[H]
    \centering
    \begin{tabular}{cc|rrr}
        \toprule
                                 &        & $IoU$ prom.     & \# aciertos prom. & \% aciertos prom. \\
        AD                       & Modelo &                 &                   &                   \\
        \midrule
        \multirow[c]{2}{*}{-}    & ResNet & 0.4494          & 4                 & 22\%              \\
                                 & LeNet  & 0.4715          & 6                 & 33\%              \\\hline
        \multirow[c]{2}{*}{MDD}  & ResNet & 0.5451          & 8                 & 44\%              \\
                                 & LeNet  & 0.5801          & 9                 & 50\%              \\\hline
        \multirow[c]{2}{*}{DANN} & ResNet & 0.6941          & 12                & 67\%              \\
                                 & LeNet  & 0.7024          & 12                & 67\%              \\\hline
        \multirow[c]{2}{*}{AFN}  & ResNet & \textbf{0.7486} & \textbf{13}       & \textbf{72\%}     \\
                                 & LeNet  & 0.6493          & 11                & 61\%              \\\hline
        \multirow[c]{2}{*}{ADDA} & ResNet & 0.6763          & 11                & 61\%              \\
                                 & LeNet  & 0.6406          & 10                & 56\%              \\
        \bottomrule
    \end{tabular}
    \caption{IoU promedio y cantidad promedio de aciertos al aplicar los modelos a cada telegrama.}
    \label{tab:iou-cant-aciertos-en-telegramas}
\end{table}

El cuadro \ref{tab:iou-cant-aciertos-en-telegramas} confirma la elecci\'on del mejor modelo realizada anteriormente.
Independientemente de qu\'e t\'ecnica de adaptaci\'on se utilice, todas presentan mejores porcentajes de aciertos
promedio que los modelos que fueron entrenados \'unicamente con $MNIST$.

\begin{figure}[H]
    \centering
    \includegraphics[width=1\textwidth]{chapter4/dist-aciertos.png}
    \caption{Distribuci\'on de aciertos de cantidad de votos por cada par t\'ecnica AD y modelo.}
    \label{fig:distribucion-aciertos}
\end{figure}

\section{Comparaci\'on de espacios latentes}

\lipsum[1]

\section{\'Analisis de errores}

Los errores de predicci\'on de los modelos pueden reflejarse en las distribuci\'on de la m\'etrica $IoU$ para cada uno
de ellos.

\begin{figure}[H]
    \centering
    \begin{subfigure}[h]{0.43\textwidth}
        \includegraphics[height=1\textwidth]{chapter4/hist-iou-sin-da.png}
    \end{subfigure}
    \hfill
    \begin{subfigure}[h]{0.43\textwidth}
        \includegraphics[height=1\textwidth]{chapter4/hist-iou-mdd.png}
    \end{subfigure}
    \hfill
    \begin{subfigure}[h]{0.43\textwidth}
        \includegraphics[height=1\textwidth]{chapter4/hist-iou-dann.png}
    \end{subfigure}
    \hfill
    \begin{subfigure}[h]{0.43\textwidth}
        \includegraphics[height=1\textwidth]{chapter4/hist-iou-afn.png}
    \end{subfigure}
    \hfill
    \begin{subfigure}[h]{0.43\textwidth}
        \includegraphics[height=1\textwidth]{chapter4/hist-iou-adda.png}
    \end{subfigure}

    \caption{Histogramas de la m\'etrica $IoU$ promedio por telegrama por cada par t\'ecnica AD y modelo.}
    \label{fig:histogramas-ious}
\end{figure}

Resulta interesante mencionar existen telegramas que son m\'as dif\'iciles de analizar que otros. Esto puede
evidenciarse en los histogramas de la figura \ref{fig:histogramas-ious} donde se pueden observar un conjunto de
observaciones que contienen valores entre [0, 0.2] en todos los experimentos realizados. Luego de analizar cada uno de
estos casos, se detectaron las siguientes situaciones:

\begin{itemize}
    \item Telegramas cargados de forma err\'ones: ver ejemplo en el anexo \ref{anexo:telegrama-erroneo}.
    \item Telegramas correctos pero por alguna cuesti\'on la l\'ogica de extracci\'on de d\'igitos no funciona correctamente: ver
          ejemplo en el anexo \ref{anexo:telegrama-numeros-juntos}.
    \item Telegramas donde existen otros caracteres distintos a n\'umeros: al ser un cuadro de texto libre sin formato, los jefes
          de mesa pueden escribir lo que deseen. Ver ejemplo en el anexo \ref{anexo:telegrama-erroneo-caracteres-especiales}
          donde se representa el $0$ a la izquierda con $X$.
    \item Telegramas de mesas donde la mayor cantidad de votos se las lleva un \'unico partido y completan los votos a la
          izquierda con $0$: al agregar los ceros a la izquierda, aumenta la probabilidad de que el modelo se equivoque con esos
          ceros que no aportan al n\'umero final. Ver ejemplo en el anexo \ref{anexo:telegrama-erroneo-muchos-ceros}.
\end{itemize}

En los primeros dos puntos se describen problemas problemas que fueron detectados en el proceso de ETL del cap\'itulo
\ref{Chapter3}. Estandarizar los telegramas agregando un casillero por cada d\'igito junto a mejorar el proceso de
extracci\'on de los mismos, supondr\'a una mejora considerable en las capacidades predictivas de los modelos.

El tercer punto presenta un problema dentro de la adaptaci\'on de dominio. La misma supone que, si bien los datasets de
origen y destino son diferentes pero representan lo mismo, debe existir la misma cantidad de clases entre origen y
destino. Al agregar uno o varios caracteres adicionales en $TDS$ (como es el ejemplo donde representaban el $0$ con una
$X$), se est\'a incumpliendo este supuesto.

El cuarto punto aumenta la probabilidad de error en los modelos debido a que el $0$ a la izquierda no aporta
significado alguno al n\'umero de la cantidad de votos que se desea predecir.

Estandarizar la carga de los telegramas por parte de los jefes de mesa mediante alguna capacitaci\'on permitir\'ia
reducir los errores de los puntos tres y cuatro en elecciones futuras.


\chapter{Conclusiones}

\label{Chapter5}

\section{Conclusiones}

TODO: completar. \lipsum[1]

\section{Trabajos Futuros}

TODO: completar. \lipsum[1]

%----------------------------------------------------------------------------------------
%	THESIS CONTENT - APPENDICES
%----------------------------------------------------------------------------------------

\appendix % Cue to tell LaTeX that the following "chapters" are Appendices

\chapter{Anexo: Telegramas}

\section{Ejemplo de telegrama}
\label{anexo:telegrama}

\includegraphics[height=0.8\textheight]{appendices/telegrama.jpg}

\section{Ejemplo de telegrama mal cargado}
\label{anexo:telegrama-erroneo}

\includegraphics[height=0.8\textheight]{appendices/telegrama-erroneo.jpg}

\section{Ejemplo de telegrama con n\'umeros sin espacio entre ellos}
\label{anexo:telegrama-numeros-juntos}

\includegraphics[height=0.8\textheight]{appendices/telegrama-numeros-juntos.jpg}

\section{Ejemplo de telegrama con caracteres distintos a n\'umeros}
\label{anexo:telegrama-erroneo-caracteres-especiales}

\includegraphics[height=0.8\textheight]{appendices/telegrama-erroneo-caracteres-especiales.png}

%\include{appendices/appendixB}
%\include{appendices/appendixC}

%----------------------------------------------------------------------------------------
%	BIBLIOGRAPHY
%----------------------------------------------------------------------------------------

\printbibliography[heading=bibintoc]

%----------------------------------------------------------------------------------------

\end{document}

