\chapter{Introducción}

\label{Chapter1}

La intro la podes pensar en 3 subsecciones: descripción de elecciones y contabilización de las actas hoy, la
oportunidad que se presenta con la digitalización (y los desafíos que aparecen) y luego los objetivos. Tiene que quedar
claro que problema ves y qué propones hacer a grandes rasgos. También es importante que estos objetivos se vean
reflejados en los resultados y conclusiones (es decir, prometer algo que después cumplis al final del doc)

\section{Contexto}

En Argentina se celebran elecciones cada 2 años a excepción de las presidenciales que se realizan cada 4 años. Existen,
principalmente, tres tipos de elecciones:

\begin{itemize}
    \item Elecciones nacionales, para elegir a las autoridades federales del país: el Poder Ejecutivo, constituido por el
          Presidente y el vicepresidente y el Congreso Nacional, formado por Senadores y Diputados.
    \item Elecciones provinciales y de la Ciudad de Buenos Aires o locales, para elegir a las autoridades de cada provincia: los
          poderes ejecutivos de las provincias y sus legislaturas.
    \item Elecciones municipales, regidas por las leyes y procedimientos de cada provincia.
\end{itemize}

Si bien emitir el sufragio es diferente en cada una de ellas, generalmente consta de ingresar a un cuarto oscuro,
elegir el candidato que se desea y depositar el voto en una urna. Al finalizar la jornada, las autoridades de mesas
recuentan los votos y llenan una planilla a mano alzada donde se resume la cantidad de votos obtenidos por cada
candidato o partido político. Dicha planilla es escaneada y enviada a traves de un telegrama del correo argentino al
centro de cómputo para su procesamiento. Una vez allí, se contabilizan en un sistema informático a partir de un grupo
de personas. Este proceso cuenta con una etapa de de digitalización y otra de validación (TODO: esto lo vi en el
diagrama de las elecciones de cordoba. no encontre nada oficial al respecto.).

\begin{figure}[H]
    \centering
    \includegraphics[width=1\textwidth]{chapter1/proceso-elecciones.png}
    \caption{Proceso eleccionario. TODO: cambiar para que se pueda leer o eliminar directament?}
    \label{fig:proceso-elecciones}
\end{figure}

Para que el proceso sea lo mas rápido y eficaz posible, se requiere a una gran cantidad personas destinadas al centro
de cómputo. Tal solución hace que el proceso sea altamente ineficiente en cuanto a tiempos y costos se refiere. En las
elecciones legislativas del 2021 se gastaron unos \$17.000 millones de pesos de los cuales \$4.000 millones de pesos
fueron destinados a sueldos para el personal\footnote{Fuente:
    \href{https://www.cronista.com/economia-politica/Elecciones-legislativas-2021-cuanto-mas-se-gastara-por-el-coronavirus-segun-el-Presupuesto-20201004-0006.html}{El
        cronista}}.

\section{Motivación}

Digitalizar los telegramas de manera automática supondrá un ahorro considerable en el presupuesto de las elecciones,
agilizará la obtención de los resultados y aportará transparencia al proceso en general. Es posible entrenar un modelo
de clasificación de dígitos a un costo extremadamente menor al actual y utilizarlo al momento de la contabilización de
los votos.

Contar con una digitalización automática permitirá bajar los costos debido a que se necesitará un grupo menor de
personas en el centro de cómputo. Además, el trabajo a realizar será mas simple ya que sólo constará del proceso de
validación.

La clasificación de números es un problema que, si bien parece resuelto con \cite{lecun1998gradient} y la creación del
dataset MNIST, no debe ser tomada a la ligera. No existe una única forma de escribir y año a año cambian las personas
que son los jefes de mesa encargados de completar los telegramas. Las características de los números escritos a mano
difiere entre cada elección. Cuando la distribución de los datos de entrenamiento difiere a la de los datos de
aplicación, se está ante un {\it corrimiento de dominio} ({\it domain shift} o {\it data drift} en inglés). Esto
implica que un modelo que se entrene en algún dataset estático como el MNIST que fue creado en el
\citeyear{lecun1998gradient}, hará malas clasficaciones los dígitos de las elecciones.

Cuando los dominios de entrenamiento (origen) y aplicación (destino) son distintos, se debe a que hay un {\it sesgo} en
los datos. Las técnicas de entrenamiento clásicas suponen que, si bien existe el sesgo existe, éste será el mismo entre
origen y destino. Cuando el supuesto no se cumple, las predicciones del modelo se ven afectadas negativamente. El
cuadro \ref{tab:lenet-distintos-datasets} muestra un ejemplo de la degradación de la precisión cuando un modelo se
aplica a un conjunto de datos que es diferente al cual se entrenó.

\begin{table}[ht]
    \centering
    \begin{tabular}{c|ccc}
        \toprule
        \multirow{2}{*}{\diagbox[height=1.2cm, width=3cm]{Train}{Test}} & MNIST                                             & USPS                              & SVHN                              \\
                                                                        & \includegraphics[width=16px]{chapter1/mnist3.png}
        \includegraphics[width=16px]{chapter1/mnist6.png}
        \includegraphics[width=16px]{chapter1/mnist8.png}               & \includegraphics[width=16px]{chapter1/usps3.png}
        \includegraphics[width=16px]{chapter1/usps6.png}
        \includegraphics[width=16px]{chapter1/usps8.png}                & \includegraphics[width=16px]{chapter1/svhn3.png}
        \includegraphics[width=16px]{chapter1/svhn6.png}
        \includegraphics[width=16px]{chapter1/svhn8.png}                                                                                                                                            \\
        \midrule
        MNIST                                                           & \multirow{2}{*}{\textbf{99.17\%}}                 & \multirow{2}{*}{78.08\%}          & \multirow{2}{*}{31.50\%}          \\
        \includegraphics[width=16px]{chapter1/mnist3.png}
        \includegraphics[width=16px]{chapter1/mnist6.png}
        \includegraphics[width=16px]{chapter1/mnist8.png}               &                                                   &                                   &                                   \\
        USPS                                                            & \multirow{2}{*}{57.10\%}                          & \multirow{2}{*}{\textbf{95.42\%}} & \multirow{2}{*}{26.94\%}          \\
        \includegraphics[width=16px]{chapter1/usps3.png}
        \includegraphics[width=16px]{chapter1/usps6.png}
        \includegraphics[width=16px]{chapter1/usps8.png}                &                                                   &                                   &                                   \\
        SVHN                                                            & \multirow{2}{*}{61.92\%}                          & \multirow{2}{*}{64.28\%}          & \multirow{2}{*}{\textbf{89.52\%}} \\
        \includegraphics[width=16px]{chapter1/svhn3.png}
        \includegraphics[width=16px]{chapter1/svhn6.png}
        \includegraphics[width=16px]{chapter1/svhn8.png}                &                                                   &                                   &                                   \\
        \bottomrule
    \end{tabular}
    \caption{Precisión que se obtiene al aplicar un modelo LeNet-5 entrenado con otros datasets de dígitos.}
    \label{tab:lenet-distintos-datasets}
\end{table}

Es por esto que se debe recurrir a técnicas de entrenamiento más complejas donde se intenta que el modelo aprenda a sin
el sesgo de los datos. Es decir, que pueda aprender a {\it adaptarse} de un dominio a otro. En la actualidad no existen
trabajos publicados relacionados a la digitalización de telegramas en Argentina que compare cúal es la mejor técnica de
adaptación de dominio.

\section{Objetivos}

La presente tesis enfocará en el desarrollo de un modelo que permita digitalizar los telegramas de las elecciones en
Argentina utilizando las legislativas de la provincia de Santa Fe del año 2021 como comparativa. Los telegramas son
públicos y se encuentran subidos en la \href{https://op.elecciones.gob.ar/telegramas/generales2021/}{página oficial del
    estado argentino}. En el anexo \ref{anexo:telegrama} se adjunta un ejemplo de uno de ellos.

Para poder llevarlo a cabo, se procede a:
\begin{itemize}
    \item Armar el proceso de ETL de los telegramas que permita limpiar y extraer los dígitos.
    \item Determinar un conjunto de datos etiquetado a ser utilizado como dominio de origen.
    \item Entrenar distintos arquitecturas de redes convolucionales mediante técnicas de adaptación de dominio.
    \item Analizar y evaluar las métricas que permitan seleccionar el mejor modelo.
    \item Seleccionar el mejor par modelo - técnica de adaptación de dominio.
    \item Detectar oportunidades de mejora en el proceso eleccionario respecto a los telegramas.
\end{itemize}

\section{Estructura del trabajo}
Este trabajo se encuentra organizado con la siguiente estructura:

\begin{itemize}
    \item Capítulo 2: Marco teórico del reconocimiento de dígitos, redes neuronales, aprendizaje por transferencia y adaptación
          de dominio.
    \item Capítulo 3: Metodología del trabajo realizado sobre los telegramas de las elecciones de Santa Fe. Proceso de
          extracción, transformación y limpieza de los mismos. Diseño experimental y métricas de evaluación.
    \item Capítulo 4: Análisis de resultados. Análisis de métricas, espacios latentes obtenidos y errores.
    \item Capítulo 5: Conclusiones del trabajo, mejoras planteadas y futuras investigaciones.
\end{itemize}