\chapter{Conclusiones}

\label{Chapter5}

El presente trabajo se centró en la utilización de técnicas de adaptación de dominio para mejorar la digitalización de
telegramas de elecciones mediante redes neuronales. Con el fin de lograr este objetivo, se evaluaron diversos modelos
de redes neuronales entrenados con distintos algoritmos de adaptación de dominio.

Tras realizar múltiples experimentos, se pudo determinar que la red ResNet entrenada con un AFN es el modelo que
obtiene la mejor precisión en la tarea de digitalización de telegramas. Este resultado indica que el uso de modelos más
complejos y profundos puede ser beneficioso en este tipo de tarea. Sin embargo, también se pudo comprobar que una red
más simple, como la LeNet con AFN, logra un desempeño satisfactorio, lo que sugiere que no siempre es necesario
utilizar modelos complejos para resolver tareas como ésta.

En cuanto a la precisión de las predicciones en los telegramas, se logró alcanzar una precisión promedio del 73\% sin
estandarizar la casilla donde se escriben los dígitos. Podría disponerse de un casillero por número a escribir y
capacitar para no agregar ceros a la izquierda y se espera que el resultado sería aún mejor, ya que el ETL de
segmentación de dígitos sería de mejor calidad.

Es importante destacar que la utilización de técnicas de adaptación de dominio en la tarea de digitalización de
telegramas de elecciones es un campo de investigación en desarrollo, y aún existen diversas limitaciones que deben ser
abordadas. No obstante, los resultados obtenidos en este trabajo son prometedores y sugieren que la adaptación de
dominio puede ser una técnica útil para mejorar la digitalización de telegramas de elecciones en contextos de alta
variabilidad.

Cabe señalar que los resultados obtenidos en este trabajo pueden tener implicaciones importantes en la automatización
de procesos electorales, lo que puede mejorar la eficiencia y la transparencia de las elecciones. Con la digitalización
de telegramas de manera efectiva, es posible mejorar la velocidad y la precisión del conteo de votos, lo que puede
reducir la posibilidad de errores humanos y aumentar la confianza en los procesos electorales.

\section{Trabajos Futuros}

A pesar de los resultados alentadores obtenidos en este trabajo, existen diversas posibilidades de mejora y trabajos
futuros que pueden llevarse a cabo para ampliar el alcance y la calidad de los resultados. A continuación, se detallan
algunas posibles líneas de investigación futura:

\begin{itemize}
    \item Mejora del proceso ETL: Una de las limitaciones de la presente investigación es la calidad del conjunto de datos TDS,
          ya que presenta ciertas limitaciones en términos de segmentación de los dígitos en los telegramas. Por lo tanto, una
          posible línea de investigación futura es la mejora del proceso ETL para mejorar la segmentación de los dígitos y
          mejorar la calidad del conjunto de datos.
    \item Mejorar el proceso de post-procesamiento de las predicciones: En la implementación actual, no se realizó ningún tipo de
          post-procesamiento en las predicciones obtenidas por los modelos de aprendizaje automático. Sería útil explorar
          distintas opciones de post-procesamiento, como la validación de que la cantidad total de votos predichos por telegrama
          no supere el total que debe haber por mesa, o la validación de que la cantidad de votos por partido no supere el total
          de la mesa. También se podría considerar la utilización de otras técnicas de post-procesamiento, como la corrección de
          errores en las predicciones mediante la incorporación de información contextual adicional.
    \item Mejora de la estandarización de los telegramas: Como se mencionó en las conclusiones, la estandarización en la sección
          donde se escriben los dígitos podría mejorar aún más la precisión de los modelos de ya que mejoraría la calidad de TDS.
    \item Evaluación de otras técnicas de adaptación de dominio: En este trabajo se evaluaron algunas técnicas de adaptación de
          dominio, pero existen muchas otras técnicas que podrían ser útiles para mejorar la precisión de la digitalización de
          telegramas. Por lo tanto, una posible línea de investigación futura es la evaluación de otras técnicas de adaptación de
          dominio para comparar su efectividad con las técnicas utilizadas en esta investigación.
    \item Evaluación de modelos híbridos: Otra posible línea de investigación futura es la evaluación de modelos híbridos que
          combinen técnicas de adaptación de dominio con otros métodos de mejora de la precisión en la tarea de digitalización de
          telegramas. Por ejemplo, se podría explorar la combinación de redes neuronales con técnicas de procesamiento de
          imágenes para mejorar la calidad de los datos.
\end{itemize}

En conclusión, existen varias líneas de investigación desprendidas del trabajo actual que pueden ser exploradaspara
mejorar la precisión de los modelos que deben aplicarse en un conjunto de datos con características distintas que del
que se entrenó. Estas mejoras pueden ser importantes para aumentar la confianza en los procesos electorales y mejorar
la transparencia y la eficiencia en el conteo de votos.